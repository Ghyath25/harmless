\section{Conclusion}

Les objectifs du langages ont �t� bien identifi�s, tant sur la partie jeu d'instruction, que la mod�lisation interne d'un pipeline simple. 

Au niveau du jeu d'instruction, les bases du langages ont �t� d�finies et la description de plusieurs processeurs (ARM et XGate pour l'instant) a �t� r�alis�e. Il sera certainement n�cessaire d'enrichir encore le langage pour d'autre architectures. les outils sont actuellement en cours de d�veloppement, avec pour premier objectif de g�n�rer un simulateur de jeu d'instruction � partir de la description. Il faudra, dans un deuxi�me temps, interfacer cet outil avec le mod�le de l'architecture interne. 

Au niveau de l'architecture interne, deux outils permettant la g�n�ration de code � partir de la description d'un pipeline simple ont �t� d�velopp�: un premier permettant de g�n�rer le mod�le du pipeline sous la forme d'un automate � partir de la description (\emph{p2a}), et un second qui g�n�re le code source du mod�le du pipeline � partir de l'automate (\emph{a2cpp}). Ces outils ont permis de valider l'approche par automate. Il est maintenant n�cessaire d'enrichir les descriptions pour pouvoir mod�liser des processeurs plus complexes (superscalaires, avec ex�cution ordonn�e ou non), d'aborder la mod�lisation hi�rarchie m�moire, ainsi que les composants externes (timers, I/O, ...).
