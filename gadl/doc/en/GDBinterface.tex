%!TEX root = ./main.tex
%!TEX encoding = UTF-8 Unicode
\chapter{GDB interface}
\section{Introduction}
\h{} offers three ways of acting on the simulator : two APIs (C and Python) and
the GDB interface. The interface allows to use any tool that have been built on
GDB : IDE integration (eclipse) and GUI like {\tt ddd} for example.

When built for GDB, the simulator acts like a GDB server on which any GDB client
(intended for your target) can connect. The simulated program can be debugged as
any program.

In order to use the GDB interface, two things are needed :
\begin{itemize}
	\item a modified description to include GDB specific informations.
	\item a version of GDB compiled to target the described platform.
\end{itemize}

\section{GDB interface description}
The GDB specific informations are gathered in a component (see section
\ref{sec:component}). The default component must be modified to specify
which component is the GDB component : use the {\tt debug} attribute.
\begin{lstlisting}
	 debug := avrgdb
\end{lstlisting}

This component must have five methods :
\begin{lstlisting}
  u8 read8(u32 v_addr)
  void write8(u32 v_addr, u8 val)
  u8 getNBRegister()
  u32 getRegister(u8 id, out u8 sizeInBits)
  void setRegister(u8 id, u32 value)
\end{lstlisting}

Between the client and the server, GDB uses a protocol called Remote Serial
Protocol\footnote{\url{http://sourceware.org/gdb/current/onlinedocs/gdb_37.html\#SEC673}}.
It specify how data and instructions are transfered between the server and the
client. Registers are identified by a number and memories are mapped in only one
memory space. These methods link this protocol and the description and are
called by the code imitating the GDB server.

\begin{lstlisting}
  u8 read8(u32 v_addr)
\end{lstlisting}
This method has a "virtual" address as input and return the value contained at
this address. The virtual address is used by GDB to differentiate the memories
(RAM,ROM) which may have the same absolute address (on Harvard architecture for
example).
The GDB command {\tt info mem} may give the informations to write this method.

\begin{lstlisting}
  void write8(u32 v_addr, u8 val)
\end{lstlisting}
This method write {\tt val} at {\tt v\_addr} address. For example, in the AVR
model, the addresses below {\tt 0x8000000} are in the RAM, and over {\tt
0x8000000} are in the ROM.
The GDB command {\tt info mem} may give the informations to write this method.

\begin{lstlisting}
  u8 getNBRegister()
  u32 getRegister(u8 id, out u8 sizeInBits)
  void setRegister(u8 id, u32 value)
\end{lstlisting}

These methods are used by the GDB server to get/modify registers value in its order.
{\tt getNBRegister()} returns the number of register that are
transmitted between GDB client and server.
{\tt getRegister()} take an register number as
input {\tt id}, returns the value of the register and set
{\tt sizeInBits} to the value that is required by the
client. {\tt setRegister()} sets the register
{\tt id} to {\tt value}. For all these methods, the register number
is the number of the register in the order defined for the protocol between
client and server.

Some informations about these numbers may be found in  GDB sources. For the PPC
model : {\tt gdb/regformats/reg-ppc.dat} and for the AVR  model {\tt
gdb/avr-tdep.c}. The GDB command {\tt info register} or {\tt info all-registers}
may also give some useful informations on registers order and size. In order to
debug the GDB server, it can be build with the {\tt DEBUG} defined, it will dump
all the communication between client and server. For an example, look at the PPC
or AVR model in the SVN of Harmless.

\section{Short introduction to remote debugging on the simulator}

First, generate the sources and compile the GDB interface, with {\tt make gdb}.

Then launch the GDB server with {\tt <model\_name>\_gdb -f <executable> -p
<port>}. The program is started and stopped before the first instruction,
waiting for GDB orders. Then in GDB:
\begin{verbatim}
# Sometimes architecture must be set in GDB
set architecture powerpc:750

target remote :<port> # Connect to the server
info register         # Display the value of common registers
b my_function         # Add a breakpoint in my_function()
c                     # continu the simulation
# ... Simulation breaks at my_function()
bt                    # Display backtrace
print i               # Display the value of the i variable
x 42                  # examine memory at address 42
\end{verbatim}

More information about GDB command, are available in the GDB
manual\footnote{\url{http://sourceware.org/gdb/current/onlinedocs/gdb_4.html\#SEC14}}
