\chapter{Micro-architecture description}

%La description de la micro-architecture est nécessaire pour la génération d'un simulateur précis au cycle près (en anglais \og Cycle-Accurate Simulator\fg : CAS) permettant de reproduire le comportement temporel du processeur.
%
%\section{Architecture générale}
%La description de la vue \emph{micro-architecture} se fait à travers trois sous-vues: 
%\begin{itemize}
%\item La sous-vue {\tt architecture}: qui permet de décrire les différentes contraintes sur l'utilisation des composants matériels (les registres, la mémoire,\dots);
%\item La sous-vue {\tt machine}: qui sert à décrire l'architecture pipelinée, elle peut se composer d'un ou plusieurs pipelines (voir section \ref{sec:pipeSplit} pour ce dernier cas). Son rôle est de factoriser plusieurs pipelines; de ce fait, elle aurait pu être omise lorsque nous avons un seul pipeline;
%\item La sous-vue {\tt pipeline}: qui permet de décrire un pipeline.
%\end{itemize}